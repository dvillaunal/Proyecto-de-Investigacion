% Options for packages loaded elsewhere
\PassOptionsToPackage{unicode}{hyperref}
\PassOptionsToPackage{hyphens}{url}
%
\documentclass[
]{article}
\usepackage{amsmath,amssymb}
\usepackage{lmodern}
\usepackage{iftex}
\ifPDFTeX
  \usepackage[T1]{fontenc}
  \usepackage[utf8]{inputenc}
  \usepackage{textcomp} % provide euro and other symbols
\else % if luatex or xetex
  \usepackage{unicode-math}
  \defaultfontfeatures{Scale=MatchLowercase}
  \defaultfontfeatures[\rmfamily]{Ligatures=TeX,Scale=1}
\fi
% Use upquote if available, for straight quotes in verbatim environments
\IfFileExists{upquote.sty}{\usepackage{upquote}}{}
\IfFileExists{microtype.sty}{% use microtype if available
  \usepackage[]{microtype}
  \UseMicrotypeSet[protrusion]{basicmath} % disable protrusion for tt fonts
}{}
\makeatletter
\@ifundefined{KOMAClassName}{% if non-KOMA class
  \IfFileExists{parskip.sty}{%
    \usepackage{parskip}
  }{% else
    \setlength{\parindent}{0pt}
    \setlength{\parskip}{6pt plus 2pt minus 1pt}}
}{% if KOMA class
  \KOMAoptions{parskip=half}}
\makeatother
\usepackage{xcolor}
\usepackage[margin=1in]{geometry}
\usepackage{color}
\usepackage{fancyvrb}
\newcommand{\VerbBar}{|}
\newcommand{\VERB}{\Verb[commandchars=\\\{\}]}
\DefineVerbatimEnvironment{Highlighting}{Verbatim}{commandchars=\\\{\}}
% Add ',fontsize=\small' for more characters per line
\usepackage{framed}
\definecolor{shadecolor}{RGB}{248,248,248}
\newenvironment{Shaded}{\begin{snugshade}}{\end{snugshade}}
\newcommand{\AlertTok}[1]{\textcolor[rgb]{0.94,0.16,0.16}{#1}}
\newcommand{\AnnotationTok}[1]{\textcolor[rgb]{0.56,0.35,0.01}{\textbf{\textit{#1}}}}
\newcommand{\AttributeTok}[1]{\textcolor[rgb]{0.77,0.63,0.00}{#1}}
\newcommand{\BaseNTok}[1]{\textcolor[rgb]{0.00,0.00,0.81}{#1}}
\newcommand{\BuiltInTok}[1]{#1}
\newcommand{\CharTok}[1]{\textcolor[rgb]{0.31,0.60,0.02}{#1}}
\newcommand{\CommentTok}[1]{\textcolor[rgb]{0.56,0.35,0.01}{\textit{#1}}}
\newcommand{\CommentVarTok}[1]{\textcolor[rgb]{0.56,0.35,0.01}{\textbf{\textit{#1}}}}
\newcommand{\ConstantTok}[1]{\textcolor[rgb]{0.00,0.00,0.00}{#1}}
\newcommand{\ControlFlowTok}[1]{\textcolor[rgb]{0.13,0.29,0.53}{\textbf{#1}}}
\newcommand{\DataTypeTok}[1]{\textcolor[rgb]{0.13,0.29,0.53}{#1}}
\newcommand{\DecValTok}[1]{\textcolor[rgb]{0.00,0.00,0.81}{#1}}
\newcommand{\DocumentationTok}[1]{\textcolor[rgb]{0.56,0.35,0.01}{\textbf{\textit{#1}}}}
\newcommand{\ErrorTok}[1]{\textcolor[rgb]{0.64,0.00,0.00}{\textbf{#1}}}
\newcommand{\ExtensionTok}[1]{#1}
\newcommand{\FloatTok}[1]{\textcolor[rgb]{0.00,0.00,0.81}{#1}}
\newcommand{\FunctionTok}[1]{\textcolor[rgb]{0.00,0.00,0.00}{#1}}
\newcommand{\ImportTok}[1]{#1}
\newcommand{\InformationTok}[1]{\textcolor[rgb]{0.56,0.35,0.01}{\textbf{\textit{#1}}}}
\newcommand{\KeywordTok}[1]{\textcolor[rgb]{0.13,0.29,0.53}{\textbf{#1}}}
\newcommand{\NormalTok}[1]{#1}
\newcommand{\OperatorTok}[1]{\textcolor[rgb]{0.81,0.36,0.00}{\textbf{#1}}}
\newcommand{\OtherTok}[1]{\textcolor[rgb]{0.56,0.35,0.01}{#1}}
\newcommand{\PreprocessorTok}[1]{\textcolor[rgb]{0.56,0.35,0.01}{\textit{#1}}}
\newcommand{\RegionMarkerTok}[1]{#1}
\newcommand{\SpecialCharTok}[1]{\textcolor[rgb]{0.00,0.00,0.00}{#1}}
\newcommand{\SpecialStringTok}[1]{\textcolor[rgb]{0.31,0.60,0.02}{#1}}
\newcommand{\StringTok}[1]{\textcolor[rgb]{0.31,0.60,0.02}{#1}}
\newcommand{\VariableTok}[1]{\textcolor[rgb]{0.00,0.00,0.00}{#1}}
\newcommand{\VerbatimStringTok}[1]{\textcolor[rgb]{0.31,0.60,0.02}{#1}}
\newcommand{\WarningTok}[1]{\textcolor[rgb]{0.56,0.35,0.01}{\textbf{\textit{#1}}}}
\usepackage{graphicx}
\makeatletter
\def\maxwidth{\ifdim\Gin@nat@width>\linewidth\linewidth\else\Gin@nat@width\fi}
\def\maxheight{\ifdim\Gin@nat@height>\textheight\textheight\else\Gin@nat@height\fi}
\makeatother
% Scale images if necessary, so that they will not overflow the page
% margins by default, and it is still possible to overwrite the defaults
% using explicit options in \includegraphics[width, height, ...]{}
\setkeys{Gin}{width=\maxwidth,height=\maxheight,keepaspectratio}
% Set default figure placement to htbp
\makeatletter
\def\fps@figure{htbp}
\makeatother
\setlength{\emergencystretch}{3em} % prevent overfull lines
\providecommand{\tightlist}{%
  \setlength{\itemsep}{0pt}\setlength{\parskip}{0pt}}
\setcounter{secnumdepth}{-\maxdimen} % remove section numbering
\usepackage{booktabs}
\usepackage{longtable}
\usepackage{array}
\usepackage{multirow}
\usepackage{wrapfig}
\usepackage{float}
\usepackage{colortbl}
\usepackage{pdflscape}
\usepackage{tabu}
\usepackage{threeparttable}
\usepackage{threeparttablex}
\usepackage[normalem]{ulem}
\usepackage{makecell}
\usepackage{xcolor}
\ifLuaTeX
  \usepackage{selnolig}  % disable illegal ligatures
\fi
\IfFileExists{bookmark.sty}{\usepackage{bookmark}}{\usepackage{hyperref}}
\IfFileExists{xurl.sty}{\usepackage{xurl}}{} % add URL line breaks if available
\urlstyle{same} % disable monospaced font for URLs
\hypersetup{
  pdftitle={Analisís de BD de los Aspirantes UNAL \& UdeA},
  pdfauthor={Daniel Villa - Ronald Palencia},
  hidelinks,
  pdfcreator={LaTeX via pandoc}}

\title{Analisís de BD de los Aspirantes UNAL \& UdeA}
\author{Daniel Villa - Ronald Palencia}
\date{}

\begin{document}
\maketitle

\hypertarget{introducciuxf3n}{%
\section{Introducción:}\label{introducciuxf3n}}

Este archivo se creo para el tratamiento de los datos recolectados los
días 23 al 25 de octubre de los aspirantes universitarios de la UNAL
sede Medellín y UdeA.

\hypertarget{tipo-de-muestreo}{%
\section{Tipo de muestreo}\label{tipo-de-muestreo}}

se aplico un muestreo sistemático de 1 en 20, tomando como valor
aleatorio inicial por medio del software R

\begin{Shaded}
\begin{Highlighting}[]
\CommentTok{\# fijar semilla:}
\FunctionTok{set.seed}\NormalTok{(}\DecValTok{123}\NormalTok{)}

\CommentTok{\# muestreo 1 en 20 en la UdeA}
\NormalTok{udea }\OtherTok{\textless{}{-}} \FunctionTok{sample}\NormalTok{(}\DecValTok{1}\SpecialCharTok{:}\DecValTok{20}\NormalTok{,}\DecValTok{1}\NormalTok{, }\AttributeTok{replace =}\NormalTok{ F)}


\CommentTok{\# muestreo 1 en 20 en la unal}
\NormalTok{unal }\OtherTok{\textless{}{-}} \FunctionTok{sample}\NormalTok{(}\DecValTok{1}\SpecialCharTok{:}\DecValTok{10}\NormalTok{,}\DecValTok{1}\NormalTok{, }\AttributeTok{replace =}\NormalTok{ F)}

\FunctionTok{paste}\NormalTok{(}\StringTok{"UdeA = Valor generado aletoriamente: "}\NormalTok{, udea)}
\end{Highlighting}
\end{Shaded}

\begin{verbatim}
## [1] "UdeA = Valor generado aletoriamente:  15"
\end{verbatim}

\begin{Shaded}
\begin{Highlighting}[]
\FunctionTok{paste}\NormalTok{(}\StringTok{"UNAL = Valor generado aletoriamente: "}\NormalTok{, unal)}
\end{Highlighting}
\end{Shaded}

\begin{verbatim}
## [1] "UNAL = Valor generado aletoriamente:  3"
\end{verbatim}

el valor \texttt{x\ =\ \{unal,\ udea\}} fue el que se genero por medio
de la función \texttt{sample()}.

\hypertarget{fecha-de-inicio-y-finalizaciuxf3n}{%
\section{Fecha de inicio y
finalización}\label{fecha-de-inicio-y-finalizaciuxf3n}}

\hypertarget{primer-duxeda}{%
\subsection{Primer día}\label{primer-duxeda}}

\begin{quote}
Nota: Unico día de recolección de datos en la UNAL, el 24 y 25 de Oct
fueron tomados en la UdeA.
\end{quote}

las encuestas empezaron el día 23 de octubre a las 10 AM de la mañana,
terminando a las 2 PM de la tarde

\hypertarget{segundo-duxeda}{%
\subsection{Segundo día}\label{segundo-duxeda}}

las encuestas empezaron el día 24 de octubre a las 8 AM de la mañana,
terminando a las 12 del medio día

\hypertarget{tercer-duxeda}{%
\subsection{Tercer día}\label{tercer-duxeda}}

las encuestas empezaron el día 25 de octubre a las 8 AM de la mañana,
terminando a las 11 AM (esto debido a la dificultad de tomar los datos
en la UdeA).

Se leen los datos tal cual se descargan de google forms, por ende
utilizamos \texttt{clean\_names()} para organizar y estandarizar los
nombres de las variables

\begin{Shaded}
\begin{Highlighting}[]
\NormalTok{data }\OtherTok{=} \FunctionTok{read\_xlsx}\NormalTok{(}\StringTok{"unal\_udea.xlsx"}\NormalTok{, }\AttributeTok{col\_names =}\NormalTok{ T)}
\end{Highlighting}
\end{Shaded}

\begin{verbatim}
## New names:
## * `` -> `...1`
\end{verbatim}

\begin{Shaded}
\begin{Highlighting}[]
\NormalTok{data }\SpecialCharTok{\%\textless{}\textgreater{}\%} \FunctionTok{clean\_names}\NormalTok{()}
\end{Highlighting}
\end{Shaded}

ahora con las función siguiente lo que hacemos es transformar la
variable: ``año de graduación de ultimo año de secundaria''

este cambio se efectuó dado las múltiples respuestas de los usuarios
para declarar un mismo año de graduación

\begin{Shaded}
\begin{Highlighting}[]
\CommentTok{\#data$en\_que\_ano\_salio\_o\_saldra\_del\_colegio \%\textgreater{}\% unique()}

\NormalTok{data}\SpecialCharTok{$}\NormalTok{en\_que\_ano\_salio\_o\_saldra\_del\_colegio }\OtherTok{=}
  \FunctionTok{str\_replace\_all}\NormalTok{(data}\SpecialCharTok{$}\NormalTok{en\_que\_ano\_salio\_o\_saldra\_del\_colegio,}
                  \FunctionTok{c}\NormalTok{(}\StringTok{"Once"}\OtherTok{=} \StringTok{"2022"}\NormalTok{,}\StringTok{"11.0"} \OtherTok{=} \StringTok{"2022"}\NormalTok{,}
                    \StringTok{"Saldré este año"} \OtherTok{=} \StringTok{"2022"}\NormalTok{, }
                    \StringTok{"Ya sali"} \OtherTok{=} \StringTok{"2021"}\NormalTok{, }\StringTok{"En el 2019"} \OtherTok{=} \StringTok{"2019"}\NormalTok{,}
                    \StringTok{"comunicación audiovisual y multimedia"} \OtherTok{=} \StringTok{"2019"}\NormalTok{,}
                    \StringTok{"1°"} \OtherTok{=} \StringTok{"2022"}\NormalTok{, }
                    \StringTok{"2020.0"} \OtherTok{=} \StringTok{"2022"}\NormalTok{, }\StringTok{"2022.0"} \OtherTok{=} \StringTok{"2022"}\NormalTok{, }\StringTok{"2018.0"} \OtherTok{=} \StringTok{"2018"}\NormalTok{,}
                    \StringTok{"2019.0"} \OtherTok{=} \StringTok{"2019"}\NormalTok{, }\StringTok{"2017.0"} \OtherTok{=} \StringTok{"2017"}\NormalTok{, }\StringTok{"2021.0"} \OtherTok{=} \StringTok{"2021"}\NormalTok{))}
\end{Highlighting}
\end{Shaded}

Con esto en mente se hace la misma verificación de las demás variables,
por ende se entiende que las columnas que no aparecen son porque no se
encontraron defectos o algún motivo para su cambio \emph{(esto para
referirse a las preguntas de opción múltiple) }

Ahora por medio de una caracterización dada por la SNIES, tenemos las
siguientes categorías para la clasificación de las carreras de los
aspirantes

\begin{itemize}
\tightlist
\item
  A1 : Agronomía, Veterinaria y afines
\item
  A2: Bellas Artes
\item
  A3: Ciencias de la Educación
\item
  A4 : Ciencias de la Salud
\item
  A5: Ciencias Sociales y Humanas
\item
  A6 : Economía, Administración, Contaduría y afines
\item
  A7 : Ingeniería, Arquitectura, Urbanismo y afines
\item
  A8 : Matemáticas y Ciencias Naturales
\end{itemize}

Aquí se encuentran repartidas cada carrera que se ofrece en todas las
IES del país.

por eso con el siguiente código se convierte cada carrera en una
categoría de A del 1 al 8.

\begin{Shaded}
\begin{Highlighting}[]
\CommentTok{\# Creación de las áreas de la SNIES:}
\NormalTok{areas }\OtherTok{=} \FunctionTok{c}\NormalTok{(}\StringTok{\textquotesingle{}A1\textquotesingle{}}\NormalTok{,  }\StringTok{\textquotesingle{}A2\textquotesingle{}}\NormalTok{, }\StringTok{\textquotesingle{}A3\textquotesingle{}}\NormalTok{,}\StringTok{\textquotesingle{}A4\textquotesingle{}}\NormalTok{,}
          \StringTok{\textquotesingle{}A5\textquotesingle{}}\NormalTok{, }\StringTok{\textquotesingle{}A6\textquotesingle{}}\NormalTok{, }\StringTok{\textquotesingle{}A7\textquotesingle{}}\NormalTok{,}
          \StringTok{\textquotesingle{}A8\textquotesingle{}}\NormalTok{)}

\CommentTok{\# Estandarización para una mejor categorización}

\NormalTok{data}\SpecialCharTok{$}\NormalTok{carrera\_o\_licenciatura\_a\_la\_que\_se\_presenta }\SpecialCharTok{\%\textless{}\textgreater{}\%}
  \FunctionTok{tolower}\NormalTok{() }\SpecialCharTok{\%\textgreater{}\%} \FunctionTok{chartr}\NormalTok{(}\StringTok{"áéíóú"}\NormalTok{, }\StringTok{"aeiou"}\NormalTok{, .)}

\CommentTok{\# Clasificación de las carreras presentes en clases de A(1{-}8)}

\NormalTok{data}\SpecialCharTok{$}\NormalTok{carrera\_o\_licenciatura\_a\_la\_que\_se\_presenta }\OtherTok{\textless{}{-}}
  \FunctionTok{str\_replace\_all}\NormalTok{(data}\SpecialCharTok{$}\NormalTok{carrera\_o\_licenciatura\_a\_la\_que\_se\_presenta,}
                  \FunctionTok{c}\NormalTok{(}\StringTok{"(\^{}(ing).*)|(\^{}(ign).*)"} \OtherTok{=} \StringTok{"A7"}\NormalTok{, }\StringTok{"(\^{}(arq).*)"} \OtherTok{=}  \StringTok{"A7"}\NormalTok{,}
                    \StringTok{"(.*(agr).*)|(\^{}(zoo).*)"} \OtherTok{=} \StringTok{"A1"}\NormalTok{, }\StringTok{"(.*(vet).*)"} \OtherTok{=} \StringTok{"A1"}\NormalTok{,}
                    \StringTok{"(.*(esta).*)|(.*(mate).*)"} \OtherTok{=} \StringTok{"A8"}\NormalTok{, }
                    \StringTok{"(\^{}(fis).*)|(\^{}(astro).*)"} \OtherTok{=} \StringTok{"A8"}\NormalTok{, }
                    \StringTok{"(\^{}(eco).*)|(.*(nego).*)"} \OtherTok{=} \StringTok{"A6"}\NormalTok{,}
                    \StringTok{"(\^{}(admi).*)|(\^{}(archivis).*)"} \OtherTok{=} \StringTok{"A6"}\NormalTok{,}
                    \StringTok{"(\^{}(conta).*)"} \OtherTok{=} \StringTok{"A6"}\NormalTok{,}
                    \StringTok{"(\^{}(lic).*)"} \OtherTok{=} \StringTok{"A3"}\NormalTok{, }\StringTok{"(.*(polit).*)"} \OtherTok{=} \StringTok{"A5"}\NormalTok{,}
                    \StringTok{"(.*(hist).*)"} \OtherTok{=} \StringTok{"A5"}\NormalTok{,}
                    \StringTok{"(.*(derec).*)"} \OtherTok{=}\StringTok{"A5"}\NormalTok{, }\StringTok{"(.*(filo).*)"} \OtherTok{=} \StringTok{"A5"}\NormalTok{,}
                    \StringTok{"(.*(teolo).*)"} \OtherTok{=} \StringTok{"A5"}\NormalTok{, }
                    \StringTok{"(.*(antro).*)|(.*(perio).*)"} \OtherTok{=} \StringTok{"A5"}\NormalTok{,}
                    \StringTok{"(\^{}(pedag).*)|(\^{}(entren).*)"} \OtherTok{=} \StringTok{"A3"}\NormalTok{, }
                    \StringTok{"(.*(socio).*)|(.*(traducc).*)"} \OtherTok{=} \StringTok{"A5"}\NormalTok{,}
                    \StringTok{"(.*(arte).*)|(.*(cult).*)"} \OtherTok{=} \StringTok{"A2"}\NormalTok{, }
                    \StringTok{"(.*(multime).*)|(.*(maquilla).*)"} \OtherTok{=} \StringTok{"A2"}\NormalTok{,}
                    \StringTok{"(\^{}(pregrado).*)|(\^{}(bioingenieria).*)"} \OtherTok{=} \StringTok{"A7"}\NormalTok{,}
                    \StringTok{"(.*(softw).*)"} \OtherTok{=} \StringTok{"A7"}\NormalTok{,}
                    \StringTok{"(.*(medici).*)|(.*(salud).*)"} \OtherTok{=} \StringTok{"A4"}\NormalTok{, }
                    \StringTok{"(.*(trumenta).*)|(.*(enferm).*)"} \OtherTok{=} \StringTok{"A4"}\NormalTok{, }
                    \StringTok{"(.*(odonto).*)|(.*(cologia).*)"} \OtherTok{=} \StringTok{"A4"}\NormalTok{,}
                    \StringTok{"(.*(farma).*)"} \OtherTok{=} \StringTok{"A4"}\NormalTok{, }\StringTok{"(.*(sistema).*)"} \OtherTok{=} \StringTok{"A7"}\NormalTok{))}


\CommentTok{\# Presentación de la nueva variable con sus nuevos valores:}

\NormalTok{data }\SpecialCharTok{\%\textless{}\textgreater{}\%} \FunctionTok{rename}\NormalTok{(., }\AttributeTok{areas =}\NormalTok{ carrera\_o\_licenciatura\_a\_la\_que\_se\_presenta)}

\CommentTok{\#data$areas \%\textgreater{}\% table() \%\textgreater{}\% as.data.frame() \%\textgreater{}\% xtable()}
\end{Highlighting}
\end{Shaded}

\begin{table}[ht]
\centering
\begin{tabular}{rlr}
  \hline
 & . & Freq \\ 
  \hline
1 & A1 &  15 \\ 
  2 & A2 &  11 \\ 
  3 & A3 &   8 \\ 
  4 & A4 &  32 \\ 
  5 & A5 &  27 \\ 
  6 & A6 &  17 \\ 
  7 & A7 & 165 \\ 
  8 & A8 &  15 \\ 
   \hline
\end{tabular}
\end{table}

Ahora vemos que para la siguiente también tenemos problemas seccionando
la segunda opción del aspirante, por lo cual además de las ocho áreas de
las SNIES tenemos una nueva dado que hubieron estudiantes que decían que
no les interesa una segunda carrera.

\begin{Shaded}
\begin{Highlighting}[]
\CommentTok{\# data$si\_su\_anterior\_respuesta\_es\_si\_indica\_cual \%\textgreater{}\% unique() \%\textgreater{}\%  unique()}

\CommentTok{\# estandarización de las respuetas:}

\NormalTok{data}\SpecialCharTok{$}\NormalTok{si\_su\_anterior\_respuesta\_es\_si\_indica\_cual }\SpecialCharTok{\%\textless{}\textgreater{}\%}
  \FunctionTok{tolower}\NormalTok{() }\SpecialCharTok{\%\textgreater{}\%} \FunctionTok{chartr}\NormalTok{(}\StringTok{"áéíóú"}\NormalTok{, }\StringTok{"aeiou"}\NormalTok{, .)}


\CommentTok{\# Clasificación de las carreras como segunda opción de los aspirantes:}

\NormalTok{data}\SpecialCharTok{$}\NormalTok{si\_su\_anterior\_respuesta\_es\_si\_indica\_cual }\OtherTok{\textless{}{-}}
  \FunctionTok{str\_replace\_all}\NormalTok{(data}\SpecialCharTok{$}\NormalTok{si\_su\_anterior\_respuesta\_es\_si\_indica\_cual,}
                  \FunctionTok{c}\NormalTok{(}\StringTok{"(\^{}(ing).*)|(\^{}(ign).*)"} \OtherTok{=} \StringTok{"A7"}\NormalTok{, }\StringTok{"(\^{}(arq).*)"} \OtherTok{=}  \StringTok{"A7"}\NormalTok{,}
                    \StringTok{"(.*(agr).*)|(\^{}(zoo).*)"} \OtherTok{=} \StringTok{"A1"}\NormalTok{, }\StringTok{"(.*(vet).*)"} \OtherTok{=} \StringTok{"A1"}\NormalTok{,}
                    \StringTok{"(.*(esta).*)|(.*(mate).*)"} \OtherTok{=} \StringTok{"A8"}\NormalTok{,}
                    \StringTok{"(\^{}(fis).*)|(\^{}(astro).*)"} \OtherTok{=} \StringTok{"A8"}\NormalTok{, }
                    \StringTok{"(\^{}(eco).*)|(.*(nego).*)"} \OtherTok{=} \StringTok{"A6"}\NormalTok{,}
                    \StringTok{"(\^{}(admi).*)|(\^{}(archivis).*)"} \OtherTok{=} \StringTok{"A6"}\NormalTok{,}
                    \StringTok{"(\^{}(conta).*)"} \OtherTok{=} \StringTok{"A6"}\NormalTok{, }
                    \StringTok{"(\^{}(lic).*)"} \OtherTok{=} \StringTok{"A3"}\NormalTok{,}
                    \StringTok{"(.*(polit).*)"} \OtherTok{=} \StringTok{"A5"}\NormalTok{, }\StringTok{"(.*(hist).*)"} \OtherTok{=} \StringTok{"A5"}\NormalTok{,}
                    \StringTok{"(.*(derec).*)"} \OtherTok{=} \StringTok{"A5"}\NormalTok{, }\StringTok{"(.*(filo).*)"} \OtherTok{=} \StringTok{"A5"}\NormalTok{,}
                    \StringTok{"(.*(teolo).*)"} \OtherTok{=} \StringTok{"A5"}\NormalTok{, }
                    \StringTok{"(.*(antro).*)|(.*(perio).*)"} \OtherTok{=} \StringTok{"A5"}\NormalTok{,}
                    \StringTok{"(\^{}(pedag).*)|(\^{}(entren).*)"} \OtherTok{=} \StringTok{"A3"}\NormalTok{, }
                    \StringTok{"(.*(socio).*)|(.*(traduc).*)"} \OtherTok{=} \StringTok{"A5"}\NormalTok{,}
                    \StringTok{"(.*(arte).*)|(.*(cult).*)"} \OtherTok{=} \StringTok{"A2"}\NormalTok{, }
                    \StringTok{"(.*(multime).*)|(.*(maquilla).*)"} \OtherTok{=} \StringTok{"A2"}\NormalTok{,}
                    \StringTok{"(\^{}(pregrado).*)|(\^{}(bioingenieria).*)"} \OtherTok{=} \StringTok{"A7"}\NormalTok{,}
                    \StringTok{"(.*(softw).*)"} \OtherTok{=} \StringTok{"A7"}\NormalTok{,}
                    \StringTok{"(.*(medici).*)|(.*(salud).*)"} \OtherTok{=} \StringTok{"A4"}\NormalTok{, }
                    \StringTok{"(.*(trumenta).*)|(.*(enferm).*)"} \OtherTok{=} \StringTok{"A4"}\NormalTok{, }
                    \StringTok{"(.*(odonto).*)|(.*(cologia).*)"} \OtherTok{=} \StringTok{"A4"}\NormalTok{,}
                    \StringTok{"(.*(farma).*)"} \OtherTok{=} \StringTok{"A4"}\NormalTok{, }\StringTok{"(.*(una).*)|(.*(so).*)"} \OtherTok{=} \StringTok{"no"}\NormalTok{, }
                    \StringTok{"(.*(interesa).*)|(.*(nt).*)"} \OtherTok{=} \StringTok{"no"}\NormalTok{,}
                    \StringTok{"(.*([.]).*)|(.*(uwu).*)"} \OtherTok{=} \StringTok{"no"}\NormalTok{,}
                    \StringTok{"(.*(music).*)|(.*(audio).*)"} \OtherTok{=} \StringTok{"A2"}\NormalTok{,}
                    \StringTok{"(.*(cine).*)"} \OtherTok{=} \StringTok{"A2"}\NormalTok{,}
                    \StringTok{"(.*(biol).*)|(.*(nutri).*)"} \OtherTok{=} \StringTok{"A4"}\NormalTok{,}
                    \StringTok{"(.*(bact).*)"} \OtherTok{=} \StringTok{"A4"}\NormalTok{, }
                    \StringTok{"(.*(quimi).*)"} \OtherTok{=} \StringTok{"A8"}\NormalTok{,}
                    \StringTok{"(.*(adminis).*)"} \OtherTok{=} \StringTok{"A6"}\NormalTok{, }\StringTok{"(.*(educac).*)"} \OtherTok{=} \StringTok{"A3"}\NormalTok{,}
                    \StringTok{"(.*(web).*)|(.*(mecatro).*)"} \OtherTok{=} \StringTok{"A7"}\NormalTok{,}
                    \StringTok{"(.*(web).*)|(.*(aviaci).*)"} \OtherTok{=} \StringTok{"A7"}\NormalTok{,}
                    \StringTok{"(.*([n/a]).*)"} \OtherTok{=} \StringTok{"no"}\NormalTok{)) }

\CommentTok{\# Visualización de la segunda opción por áreas de conocimiento:}

\NormalTok{data }\SpecialCharTok{\%\textless{}\textgreater{}\%} \FunctionTok{rename}\NormalTok{(., }\AttributeTok{segunda\_opcion  =}\NormalTok{ si\_su\_anterior\_respuesta\_es\_si\_indica\_cual)}

\CommentTok{\# data$segunda\_opcion \%\textgreater{}\%  table() \%\textgreater{}\%  as.data.frame() \%\textgreater{}\% xtable()}
\end{Highlighting}
\end{Shaded}

\begin{table}[ht]
\centering
\begin{tabular}{rlr}
  \hline
 & . & Freq \\ 
  \hline
1 & A1 &  12 \\ 
  2 & A2 &   8 \\ 
  3 & A3 &   6 \\ 
  4 & A4 &  16 \\ 
  5 & A5 &  14 \\ 
  6 & A6 &  27 \\ 
  7 & A7 &  66 \\ 
  8 & A8 &   8 \\ 
  9 & no & 108 \\ 
   \hline
\end{tabular}
\end{table}

una vez tenemos nuestras variables depuradas, ahora pasamos a generar un
subset, es decir, otra base de datos con las variables que son
relevantes para el problema (se quita la marca temporal de la fecha y
hora de la encuesta).

\hypertarget{base-de-datos-final}{%
\section{Base de datos final}\label{base-de-datos-final}}

Ahora sacamos las tres primeras columnas que corresponden a datos que no
son relevantes:

\begin{itemize}
\item
  X1: numeración de la fila
\item
  marca\_temporal: fecha y hora de la realización de la encuesta
\item
  por\_favor\_acepta\_nuestra\_politica\_de\_privacidad: aceptación de
  los tratamientos de los datos.
\end{itemize}

\begin{Shaded}
\begin{Highlighting}[]
\CommentTok{\# Base de datos final:}
\NormalTok{data }\OtherTok{\textless{}{-}}\NormalTok{ data[}\SpecialCharTok{{-}}\FunctionTok{c}\NormalTok{(}\DecValTok{1}\SpecialCharTok{:}\DecValTok{3}\NormalTok{)]}
\end{Highlighting}
\end{Shaded}

ahora verificamos que todo este correcto en nuestro nuevo data set:

\begin{Shaded}
\begin{Highlighting}[]
\CommentTok{\# Cambiamos fluido por Fluido}
\NormalTok{data[}\DecValTok{199}\NormalTok{,}\DecValTok{1}\NormalTok{] }\OtherTok{\textless{}{-}} \StringTok{"Fluido"}
\NormalTok{data[}\DecValTok{16}\NormalTok{,}\DecValTok{1}\NormalTok{] }\OtherTok{\textless{}{-}} \StringTok{"Fluido"}

\CommentTok{\# Convertimos en factor las siguientes variables:}

\NormalTok{data}\SpecialCharTok{$}\NormalTok{genero\_al\_que\_pertenece }\SpecialCharTok{\%\textless{}\textgreater{}\%} \FunctionTok{factor}\NormalTok{()}

\NormalTok{data}\SpecialCharTok{$}\NormalTok{tipo\_de\_colegio\_que\_estudia\_o\_estudio }\SpecialCharTok{\%\textless{}\textgreater{}\%} \FunctionTok{factor}\NormalTok{()}

\NormalTok{data}\SpecialCharTok{$}\NormalTok{en\_que\_ano\_salio\_o\_saldra\_del\_colegio }\SpecialCharTok{\%\textless{}\textgreater{}\%}  \FunctionTok{factor}\NormalTok{()}

\NormalTok{data}\SpecialCharTok{$}\NormalTok{trabaja\_actualmente }\SpecialCharTok{\%\textless{}\textgreater{}\%}  \FunctionTok{factor}\NormalTok{()}

\NormalTok{data}\SpecialCharTok{$}\NormalTok{si\_su\_anterior\_respuesta\_es\_si\_su\_trabajo\_esta\_relacionado\_con\_la\_carrera\_a\_la\_cual\_se\_presento }\SpecialCharTok{\%\textless{}\textgreater{}\%}  \FunctionTok{factor}\NormalTok{()}


\NormalTok{data}\SpecialCharTok{$}\NormalTok{areas }\SpecialCharTok{\%\textless{}\textgreater{}\%} \FunctionTok{factor}\NormalTok{()}


\NormalTok{data}\SpecialCharTok{$}\NormalTok{escoja\_una\_de\_las\_siguientes\_razones\_por\_la\_que\_va\_a\_estudiar }\SpecialCharTok{\%\textless{}\textgreater{}\%} \FunctionTok{factor}\NormalTok{()}


\NormalTok{data}\SpecialCharTok{$}\NormalTok{seleccione\_la\_principal\_razon\_para\_la\_eleccion\_de\_la\_carrera\_anterior }\SpecialCharTok{\%\textless{}\textgreater{}\%} 
  \FunctionTok{factor}\NormalTok{()}


\NormalTok{data}\SpecialCharTok{$}\NormalTok{le\_interesa\_otra\_carrera\_diferente\_a\_la\_que\_se\_presento }\SpecialCharTok{\%\textless{}\textgreater{}\%} \FunctionTok{factor}\NormalTok{()}

\NormalTok{data}\SpecialCharTok{$}\NormalTok{segunda\_opcion }\SpecialCharTok{\%\textless{}\textgreater{}\%} \FunctionTok{factor}\NormalTok{()}

\NormalTok{data}\SpecialCharTok{$}\NormalTok{la\_anterior\_carrera\_esta\_disponible\_en\_esta\_u\_otra\_universidad }\SpecialCharTok{\%\textless{}\textgreater{}\%} \FunctionTok{factor}\NormalTok{()}


\NormalTok{data}\SpecialCharTok{$}\NormalTok{cuenta\_con\_algun\_estudio\_curso\_tecnico\_entre\_otros\_certificable\_que\_este\_relacionado\_con\_la\_carrera\_que\_selecciono }\SpecialCharTok{\%\textless{}\textgreater{}\%} \FunctionTok{factor}\NormalTok{()}

\CommentTok{\# Cambiamos algunos valores para evitar más categorías.}

\NormalTok{data[}\DecValTok{144}\SpecialCharTok{:}\DecValTok{145}\NormalTok{,}\DecValTok{13}\NormalTok{] }\OtherTok{\textless{}{-}} \FunctionTok{rep}\NormalTok{(}\StringTok{"Todas las anteriores"}\NormalTok{, }\DecValTok{2}\NormalTok{)}

\CommentTok{\# Convertir a factor la ultima variable}

\NormalTok{data}\SpecialCharTok{$}\NormalTok{universidad\_a\_la\_que\_aspira }\SpecialCharTok{\%\textless{}\textgreater{}\%}  \FunctionTok{factor}\NormalTok{()}
\end{Highlighting}
\end{Shaded}

Para ver un resumen rápido tipo \texttt{summary()} con la función skim
de la librería que que lleva el mismo nombre tenemos la siguiente tabla:

\begin{table}
\centering
\resizebox{\linewidth}{!}{
\begin{tabular}{llrrlrl}
\toprule
skim\_type & skim\_variable & n\_missing & complete\_rate & factor.ordered & factor.n\_unique & factor.top\_counts\\
\midrule
factor & genero\_al\_que\_pertenece & 0 & 1.0000000 & FALSE & 4 & Fem: 150, Mas: 135, Flu: 4, Bin: 1\\
factor & tipo\_de\_colegio\_que\_estudia\_o\_estudio & 0 & 1.0000000 & FALSE & 2 & Pub: 203, Pri: 87\\
factor & en\_que\_ano\_salio\_o\_saldra\_del\_colegio & 0 & 1.0000000 & FALSE & 5 & 202: 240, 202: 38, 201: 9, 201: 2\\
factor & trabaja\_actualmente & 0 & 1.0000000 & FALSE & 2 & No: 235, Si: 55\\
factor & si\_su\_anterior\_respuesta\_es\_si\_su\_trabajo\_esta\_relacionado\_con\_la\_carrera\_a\_la\_cual\_se\_presento & 46 & 0.8413793 & FALSE & 2 & No: 209, Si: 35\\
\addlinespace
factor & areas & 0 & 1.0000000 & FALSE & 8 & A7: 165, A4: 32, A5: 27, A6: 17\\
factor & escoja\_una\_de\_las\_siguientes\_razones\_por\_la\_que\_va\_a\_estudiar & 0 & 1.0000000 & FALSE & 11 & Me : 108, Qui: 92, Qui: 20, Con: 16\\
factor & seleccione\_la\_principal\_razon\_para\_la\_eleccion\_de\_la\_carrera\_anterior & 0 & 1.0000000 & FALSE & 21 & Las: 105, Las: 55, Rec: 21, La : 20\\
factor & le\_interesa\_otra\_carrera\_diferente\_a\_la\_que\_se\_presento & 0 & 1.0000000 & FALSE & 3 & Si: 170, No: 113, Pue: 7\\
factor & segunda\_opcion & 25 & 0.9137931 & FALSE & 9 & no: 108, A7: 66, A6: 27, A4: 16\\
\addlinespace
factor & la\_anterior\_carrera\_esta\_disponible\_en\_esta\_u\_otra\_universidad & 0 & 1.0000000 & FALSE & 3 & Si: 263, No: 20, No : 7\\
factor & cuenta\_con\_algun\_estudio\_curso\_tecnico\_entre\_otros\_certificable\_que\_este\_relacionado\_con\_la\_carrera\_que\_selecciono & 0 & 1.0000000 & FALSE & 2 & No: 194, Sí: 96\\
factor & universidad\_a\_la\_que\_aspira & 12 & 0.9586207 & FALSE & 6 & Uni: 163, Uni: 105, Uni: 6, Tod: 2\\
\bottomrule
\end{tabular}}
\end{table}

\hypertarget{analisuxeds-de-nuestras-variables}{%
\section{Analisís de nuestras
variables:}\label{analisuxeds-de-nuestras-variables}}

antes de empezar el análisis tenemos que en la variable sexo esta se
registra como ``Maculino'' y con la función siguiente cambiamos el valor
a ``Masculino''.

\begin{Shaded}
\begin{Highlighting}[]
\DocumentationTok{\#\# change factor levels}
\NormalTok{data }\OtherTok{\textless{}{-}} \FunctionTok{mutate}\NormalTok{(data,}
               \AttributeTok{genero\_al\_que\_pertenece =}
                 \FunctionTok{refactor}\NormalTok{(genero\_al\_que\_pertenece,}
                          \AttributeTok{levs =} \FunctionTok{c}\NormalTok{(}\StringTok{"Binario"}\NormalTok{,}\StringTok{"Femenino"}\NormalTok{,}\StringTok{"Fluido"}\NormalTok{),}
                          \AttributeTok{repl =} \StringTok{"Masculino"}\NormalTok{))}
\end{Highlighting}
\end{Shaded}

Base de datos organizada

\begin{Shaded}
\begin{Highlighting}[]
\CommentTok{\# filter and sort the dataset}
\CommentTok{\# data \%\textgreater{}\%}
\CommentTok{\#   select(genero\_al\_que\_pertenece:universidad\_a\_la\_que\_aspira) \%\textgreater{}\% head() \%\textgreater{}\%}
\CommentTok{\#   xtable()}
\end{Highlighting}
\end{Shaded}

\begin{table}[ht]
\centering
\begin{tabular}{rlllllllllllll}
  \hline
 & genero\_al\_que\_pertenece & tipo\_de\_colegio\_que\_estudia & ano\_saldra\_del\_colegio
 & trabaja\_actualmente &
 su\_trabajo\_esta\_relacionado\_con\_la\_carrera\_a\_la\_cual\_se\_presento & areas &
 razones\_por\_la\_que\_va\_a\_estudiar &
 principal\_razon\_para\_la\_eleccion\_de\_la\_carrera & le\_interesa\_otra\_carrera &
 segunda\_opcion & disponible\_en\_esta\_u\_otra\_universidad &
 estudio\_curso\_tecnico\_entre\_otros\_certificable & universidad\_a\_la\_que\_aspira \\ 
  \hline
1 & Masculino & Publico & 2022 & No & No & A5 & Me interesa estudiar & Las asignaturas del pensum y las líneas de énfasis son de mi interés. & No & no & Si & Sí & Universidad Nacional de Colombia \\ 
  2 & Masculino & Publico & 2022 & Si & No & A3 & Quiero superarme & Recomendación de un amigo & Si & A2 & No & No & Universidad Nacional de Colombia \\ 
  3 & Masculino & Publico & 2022 & No & No & A7 & Quiero superarme & Los egresados de esta carrera consiguen empleo fácilmente & Si & A7 & Si & No & Universidad Nacional de Colombia \\ 
  4 & Masculino & Publico & 2022 & No & No & A7 & Quiero ganar bien cuando sea profesionista & Las asignaturas del pensum y las líneas de énfasis son de mi interés. & Si & A7 & Si & Sí & Universidad Nacional de Colombia \\ 
  5 & Masculino & Privado & 2022 & No & No & A6 & Me interesa estudiar & Las asignaturas del pensum y las líneas de énfasis son de mi interés. & No & no & Si & No & Universidad Nacional de Colombia \\ 
  6 & Femenino & Publico & 2021 & No & No & A7 & Quiero superarme & Las asignaturas del pensum y las líneas de énfasis son de mi interés. & No & no & Si & No & Universidad Nacional de Colombia \\ 
   \hline
\end{tabular}
\end{table}

Aquí vemos el encabezado de nuestros datos, es decir, los primes 5 datos
de nuestra base de datos de 290 observaciones.

\hypertarget{tabla-de-frecuencias-con-su-respectivo-gruxe1fico-para-una-mejor-visualizaciuxf3n}{%
\subsection{Tabla de Frecuencias, con su respectivo gráfico para una
mejor
visualización}\label{tabla-de-frecuencias-con-su-respectivo-gruxe1fico-para-una-mejor-visualizaciuxf3n}}

Tabla y gráfico del genero alusivo.

\begin{Shaded}
\begin{Highlighting}[]
\NormalTok{result }\OtherTok{\textless{}{-}} \FunctionTok{pivotr}\NormalTok{(}
\NormalTok{  data, }
  \AttributeTok{cvars =} \StringTok{"genero\_al\_que\_pertenece"}\NormalTok{, }
  \AttributeTok{nr =} \ConstantTok{Inf}
\NormalTok{)}
\CommentTok{\# summary()}
\CommentTok{\#result$tab\_freq \%\textgreater{}\% xtable()}
\end{Highlighting}
\end{Shaded}

\begin{table}[ht]
\centering
\begin{tabular}{rlr}
  \hline
 & genero\_al\_que\_pertenece & n\_obs \\ 
  \hline
1 & Binario &   1 \\ 
  2 & Femenino & 150 \\ 
  3 & Fluido &   2 \\ 
  4 & género fluido &   2 \\ 
  5 & Maculino & 135 \\ 
  6 & Total & 290 \\ 
   \hline
\end{tabular}
\end{table}

Podemos notar que en la muestras tomada (de forma sistemática de 1-20)
pudimos visualizar un mayor numero de mujeres que fueron más receptivas
al momento de realizar la encuesta, además, contamos con algunos otros
géneros que no son significativos como el Binario y el Fluido.

Para el caso del tipo de colegio donde estudió o estudia:

\begin{Shaded}
\begin{Highlighting}[]
\NormalTok{result }\OtherTok{\textless{}{-}} \FunctionTok{pivotr}\NormalTok{(}
\NormalTok{  data, }
  \AttributeTok{cvars =} \StringTok{"tipo\_de\_colegio\_que\_estudia\_o\_estudio"}\NormalTok{, }
  \AttributeTok{nr =} \ConstantTok{Inf}
\NormalTok{)}
\CommentTok{\# summary()}
\CommentTok{\#result$tab\_freq \%\textgreater{}\% xtable()}
\end{Highlighting}
\end{Shaded}

\begin{table}[ht]
\centering
\begin{tabular}{rlr}
  \hline
 & tipo\_de\_colegio\_que\_estudia\_o\_estudio & n\_obs \\ 
  \hline
1 & Privado &  87 \\ 
  2 & Publico & 203 \\ 
  3 & Total & 290 \\ 
   \hline
\end{tabular}
\end{table}

Son mayoritarios los colegio que estudian en colegios del estado,
revelando que la mayoría de los aspirantes a universidades publicas no
vienen de colegios privados.

\begin{Shaded}
\begin{Highlighting}[]
\NormalTok{result }\OtherTok{\textless{}{-}} \FunctionTok{pivotr}\NormalTok{(}
\NormalTok{  data, }
  \AttributeTok{cvars =} \StringTok{"en\_que\_ano\_salio\_o\_saldra\_del\_colegio"}\NormalTok{, }
  \AttributeTok{normalize =} \StringTok{"total"}\NormalTok{, }
  \AttributeTok{nr =} \ConstantTok{Inf}
\NormalTok{)}
\CommentTok{\# summary()}
\CommentTok{\# result$tab\_freq \%\textgreater{}\% xtable()}
\end{Highlighting}
\end{Shaded}

\begin{table}[ht]
\centering
\begin{tabular}{rlr}
  \hline
 & en\_que\_ano\_salio\_o\_saldra\_del\_colegio & n\_obs \\ 
  \hline
1 & 2017 &   1 \\ 
  2 & 2018 &   2 \\ 
  3 & 2019 &   9 \\ 
  4 & 2021 &  38 \\ 
  5 & 2022 & 240 \\ 
  6 & Total & 290 \\ 
   \hline
\end{tabular}
\end{table}

Las promociones de 2022, siempre presentan un numero mayoritario en los
aspirantes esto se debe al segundo corte o la prueba de final de año,
para el ingreso en el 2023 esto lo decimos ya que podríamos afirmar que
en le primer corte los aspirantes salieron del colegio en épocas
diferentes, más homogeneidad.

\begin{Shaded}
\begin{Highlighting}[]
\NormalTok{result }\OtherTok{\textless{}{-}} \FunctionTok{pivotr}\NormalTok{(}
\NormalTok{  data, }
  \AttributeTok{cvars =} \StringTok{"trabaja\_actualmente"}\NormalTok{, }
  \AttributeTok{nr =} \ConstantTok{Inf}
\NormalTok{)}
\CommentTok{\# summary()}
\CommentTok{\# result$tab\_freq \%\textgreater{}\% xtable()}
\end{Highlighting}
\end{Shaded}

\begin{table}[ht]
\centering
\begin{tabular}{rlr}
  \hline
 & trabaja\_actualmente & n\_obs \\ 
  \hline
1 & No & 235 \\ 
  2 & Si &  55 \\ 
  3 & Total & 290 \\ 
   \hline
\end{tabular}
\end{table}

En el segundo corte se esperan aspirantes que aun no han salido de
estudiar, por ende la mayoría no cuenta con un trabajo, podríamos
afirmar que para el primer corte del 2023 estos estudiantes que no
ingresaron posiblemente estén laborando para entonces.

\begin{Shaded}
\begin{Highlighting}[]
\NormalTok{result }\OtherTok{\textless{}{-}} \FunctionTok{pivotr}\NormalTok{(}
\NormalTok{  data, }
  \AttributeTok{cvars =} \StringTok{"areas"}\NormalTok{, }
  \AttributeTok{tabsort =} \StringTok{"areas"}\NormalTok{, }
  \AttributeTok{nr =} \ConstantTok{Inf}
\NormalTok{)}
\CommentTok{\# summary()}
\CommentTok{\# result$tab\_freq \%\textgreater{}\% xtable()}
\end{Highlighting}
\end{Shaded}

\begin{table}[ht]
\centering
\begin{tabular}{rlr}
  \hline
 & areas & n\_obs \\ 
  \hline
1 & A1 &  15 \\ 
  2 & A2 &  11 \\ 
  3 & A3 &   8 \\ 
  4 & A4 &  32 \\ 
  5 & A5 &  27 \\ 
  6 & A6 &  17 \\ 
  7 & A7 & 165 \\ 
  8 & A8 &  15 \\ 
  9 & Total & 290 \\ 
   \hline
\end{tabular}
\end{table}

Ahora una de las variables más importantes, es saber cuales son las
aspiraciones en áreas de conocimiento de los que hicieron examen de
admisión, por lo cual planteamos una vez más las diferentes áreas del
saber, en las cuales se aterrorizan todas las carreras brindadas en el
estado de Colombia:

\begin{itemize}
\item
  A1 = Agronomía, Veterinaria y afines
\item
  A2 = Bellas Artes
\item
  A3 = Ciencias de la Educación
\item
  A4 = Ciencias de la Salud
\item
  A5 = Ciencias Sociales y Humanas
\item
  A6 = Economía, Administración, Contaduría y afines
\item
  A7 = Ingeniería, Arquitectura, Urbanismo y afines
\item
  A8 = Matemáticas y Ciencias Naturales
\end{itemize}

Ahora, observando la tabla tenemos que las ingenieras, arquitectura,
urbanismo y afines, son el área más demanda de nuestra muestra.

En parte era evidente ya que gran parte de las encuestas fueron
realizadas en al UNAL sede Medellín, una universidad que la mayoría, por
no decir todas las carreras son de esta área.

\begin{Shaded}
\begin{Highlighting}[]
\NormalTok{result }\OtherTok{\textless{}{-}} \FunctionTok{pivotr}\NormalTok{(}
\NormalTok{  data, }
  \AttributeTok{cvars =} \StringTok{"escoja\_una\_de\_las\_siguientes\_razones\_por\_la\_que\_va\_a\_estudiar"}\NormalTok{, }
  \AttributeTok{tabsort =} \StringTok{"desc(n\_obs)"}\NormalTok{, }
  \AttributeTok{nr =} \ConstantTok{Inf}
\NormalTok{)}
\CommentTok{\# summary()}
\CommentTok{\# result$tab\_freq \%\textgreater{}\% xtable()}
\end{Highlighting}
\end{Shaded}

\begin{table}[ht]
\centering
\begin{tabular}{rlr}
  \hline
 & escoja\_una\_de\_las\_siguientes\_razones\_por\_la\_que\_va\_a\_estudiar & n\_obs \\ 
  \hline
1 & Así lo hacen mis parientes, mis amigos, mi novio (a) &   1 \\ 
  2 & Así lo quieren mis papás &   4 \\ 
  3 & Con los estudios que tengo no podré conseguir un buen trabajo &  16 \\ 
  4 & Es normal &   7 \\ 
  5 & Me interesa estudiar & 108 \\ 
  6 & Me va a dar prestigio. &  10 \\ 
  7 & No tengo otra cosa que hacer &  12 \\ 
  8 & Quiero ganar bien cuando sea profesionista &  14 \\ 
  9 & Quiero superarme &  92 \\ 
  10 & Quiero vivir más independiente de mi familia &  20 \\ 
  11 & Un (a) maestro (a) me dijo que siguiera estudiando &   6 \\ 
  12 & Total & 290 \\ 
   \hline
\end{tabular}
\end{table}

Una de las razones por la cual los aspirantes desean entrar a una
universidad es porque tienen un alto interés en superarse o bien en
estudiar.

\begin{Shaded}
\begin{Highlighting}[]
\NormalTok{result }\OtherTok{\textless{}{-}} \FunctionTok{pivotr}\NormalTok{(}
\NormalTok{  data, }
  \AttributeTok{cvars =} \StringTok{"seleccione\_la\_principal\_razon\_para\_la\_eleccion\_de\_la\_carrera\_anterior"}\NormalTok{, }
  \AttributeTok{tabsort =} \StringTok{"desc(n\_obs)"}\NormalTok{, }
  \AttributeTok{nr =} \ConstantTok{Inf}
\NormalTok{)}
\CommentTok{\# summary()}
\CommentTok{\# result$tab\_freq \%\textgreater{}\% xtable()}
\end{Highlighting}
\end{Shaded}

\begin{table}[ht]
\centering
\begin{tabular}{rlr}
  \hline
 & seleccione\_la\_principal\_razon\_para\_la\_eleccion\_de\_la\_carrera\_anterior & n\_obs \\ 
  \hline
1 & Esta relacionado con temáticas que me apasionan &   2 \\ 
  2 & Influencia familiar &  16 \\ 
  3 & Interés propio &   1 \\ 
  4 & La facilidad para el pago de esta carrera o para obtener becas &  20 \\ 
  5 & Las asignaturas del pensum y las líneas de énfasis son de mi interés. & 105 \\ 
  6 & Las materias de la carrera me llaman la atención &  55 \\ 
  7 & llamativa &   1 \\ 
  8 & Los egresados de esta carrera consiguen empleo fácilmente &  17 \\ 
  9 & Los egresados de esta carrera son bien remunerados o tienen buenas ofertas laborales salariales. &  16 \\ 
  10 & Me gusta &   1 \\ 
  11 & Pasión de infancia &   3 \\ 
  12 & Pasión desde la infancia &   3 \\ 
  13 & Porque me apasiona &   1 \\ 
  14 & Porque me gusta &   1 \\ 
  15 & Quiero aprender a defender la palabra de Dios con la apologetica &   3 \\ 
  16 & Quiero entender el mundo &   1 \\ 
  17 & Recomendación de un amigo &  21 \\ 
  18 & Seguir mi media tecnica y ganar dinero &   1 \\ 
  19 & Tendencia &  13 \\ 
  20 & vocación &   3 \\ 
  21 & Vocación &   6 \\ 
  22 & Total & 290 \\ 
   \hline
\end{tabular}
\end{table}

Ahora esperamos que sea cierto que la mayoría de los aspirantes hayan
leído el pensum, para decidir que esta es su linea de interés.

\begin{Shaded}
\begin{Highlighting}[]
\NormalTok{result }\OtherTok{\textless{}{-}} \FunctionTok{pivotr}\NormalTok{(}
\NormalTok{  data, }
  \AttributeTok{cvars =} \StringTok{"cuenta\_con\_algun\_estudio\_curso\_tecnico\_entre\_otros\_certificable\_que\_este\_relacionado\_con\_la\_carrera\_que\_selecciono"}\NormalTok{, }
  \AttributeTok{nr =} \ConstantTok{Inf}
\NormalTok{)}
\CommentTok{\# summary()}
\CommentTok{\# result$tab\_freq \%\textgreater{}\% xtable()}
\end{Highlighting}
\end{Shaded}

\begin{table}[ht]
\centering
\begin{tabular}{rlr}
  \hline
 & cuenta\_con\_algun\_estudio\_curso\_tecnico\_certificable & n\_obs \\ 
  \hline
  1 & No & 194 \\ 
  2 & Sí &  96 \\ 
  3 & Total & 290 \\ 
   \hline
\end{tabular}
\end{table}

Aunque creímos que los aspirantes para el 2023 no tendrían una
fundamentación en el programa que eligieron, realmente nos dejaron
sorprendidos al darnos cuenta que casi 100 personas tienen estudios que
se relacionen con su carrera.

\begin{Shaded}
\begin{Highlighting}[]
\NormalTok{result }\OtherTok{\textless{}{-}} \FunctionTok{pivotr}\NormalTok{(}
\NormalTok{  data, }
  \AttributeTok{cvars =} \StringTok{"le\_interesa\_otra\_carrera\_diferente\_a\_la\_que\_se\_presento"}\NormalTok{, }
  \AttributeTok{nr =} \ConstantTok{Inf}
\NormalTok{)}
\CommentTok{\# summary()}
\CommentTok{\# result$tab\_freq \%\textgreater{}\% xtable()}
\end{Highlighting}
\end{Shaded}

\begin{table}[ht]
\centering
\begin{tabular}{rlr}
  \hline
 & le\_interesa\_otra\_carrera\_diferente\_a\_la\_que\_se\_presento & n\_obs \\ 
  \hline
1 & No & 113 \\ 
  2 & Puede ser &   7 \\ 
  3 & Si & 170 \\ 
  4 & Total & 290 \\ 
   \hline
\end{tabular}
\end{table}

La decisión de tomar una carrera y no una linea de conocimiento, es
marcada en Colombia ya que observamos a personas cerradas en una
carrera, aunque estos mismos después eran los que esperaban al menos dos
universidades para ingresar a la misma carrera.

\begin{Shaded}
\begin{Highlighting}[]
\NormalTok{result }\OtherTok{\textless{}{-}} \FunctionTok{pivotr}\NormalTok{(}
\NormalTok{  data, }
  \AttributeTok{cvars =} \StringTok{"universidad\_a\_la\_que\_aspira"}\NormalTok{, }
  \AttributeTok{tabsort =} \StringTok{"desc(n\_obs)"}\NormalTok{, }
  \AttributeTok{nr =} \ConstantTok{Inf}
\NormalTok{)}
\CommentTok{\# summary()}
\CommentTok{\# result$tab\_freq \%\textgreater{}\% xtable()}
\end{Highlighting}
\end{Shaded}

\begin{table}[ht]
\centering
\begin{tabular}{rlr}
  \hline
 & universidad\_a\_la\_que\_aspira & n\_obs \\ 
  \hline
1 & Itm &   1 \\ 
  2 & Todas las anteriores &   2 \\ 
  3 & Universidad de Antioquia & 105 \\ 
  4 & Universidad de Medellín &   1 \\ 
  5 & Universidad Nacional de Colombia & 163 \\ 
  6 & Universidad Nacional de Colombia, Universidad de Antioquia &   6 \\ 
  7 & NA &  12 \\ 
  8 & Total & 290 \\ 
   \hline
\end{tabular}
\end{table}

Como dijimos anteriormente, la mayoría de nuestros encuestados fueron en
la UNAL sede medellín, esto se debe a que hubo un mayor cobertura al
ingresar en la universidad, lamentablemente no fue le caso de la UdeA,
la gente fue menos receptiva.

\begin{Shaded}
\begin{Highlighting}[]
\NormalTok{result }\OtherTok{\textless{}{-}} \FunctionTok{pivotr}\NormalTok{(}
\NormalTok{  data, }
  \AttributeTok{cvars =} \FunctionTok{c}\NormalTok{(}
    \StringTok{"genero\_al\_que\_pertenece"}\NormalTok{, }
    \StringTok{"areas"}
\NormalTok{  ), }
  \AttributeTok{data\_filter =} \StringTok{"genero\_al\_que\_pertenece \%in\% c(\textquotesingle{}Femenino\textquotesingle{}, \textquotesingle{}Masculino\textquotesingle{})"}\NormalTok{, }
  \AttributeTok{tabsort =} \StringTok{"desc(Masculino)"}\NormalTok{, }
  \AttributeTok{nr =} \ConstantTok{Inf}
\NormalTok{)}
\CommentTok{\# summary()}
\CommentTok{\# result$tab\_freq \%\textgreater{}\% xtable()}
\end{Highlighting}
\end{Shaded}

\begin{table}[ht]
\centering
\begin{tabular}{rlrrr}
  \hline
 & areas & Masculino & Femenino & Total \\ 
  \hline
1 & A1 &   5 &   9 &  14 \\ 
  2 & A2 &   5 &   5 &  10 \\ 
  3 & A3 &   4 &   4 &   8 \\ 
  4 & A4 &   9 &  22 &  31 \\ 
  5 & A5 &  13 &  13 &  26 \\ 
  6 & A6 &   6 &  11 &  17 \\ 
  7 & A7 &  83 &  81 & 164 \\ 
  8 & A8 &  10 &   5 &  15 \\ 
  9 & Total & 135 & 150 & 285 \\ 
   \hline
\end{tabular}
\end{table}

Tanto hombres como mujeres, prefieren las ingenierías, seguidas de las
Ciencias Sociales y Humanas.

\begin{Shaded}
\begin{Highlighting}[]
\NormalTok{result }\OtherTok{\textless{}{-}} \FunctionTok{pivotr}\NormalTok{(}
\NormalTok{  data, }
  \AttributeTok{cvars =} \FunctionTok{c}\NormalTok{(}
    \StringTok{"trabaja\_actualmente"}\NormalTok{, }
    \StringTok{"cuenta\_con\_algun\_estudio\_curso\_tecnico\_entre\_otros\_certificable\_que\_este\_relacionado\_con\_la\_carrera\_que\_selecciono"}
\NormalTok{  ), }
  \AttributeTok{data\_filter =} \StringTok{"genero\_al\_que\_pertenece \%in\% c(\textquotesingle{}Femenino\textquotesingle{}, \textquotesingle{}Masculino\textquotesingle{})"}\NormalTok{, }
  \AttributeTok{nr =} \ConstantTok{Inf}
\NormalTok{)}
\CommentTok{\# summary()}
\CommentTok{\# result$tab\_freq \%\textgreater{}\% xtable()}
\end{Highlighting}
\end{Shaded}

\begin{table}[ht]
\centering
\begin{tabular}{rlrrr}
  \hline
 & cuenta\_con\_algun\_estudio\_curso\_tecnico\_certificable & No & Si & Total \\ 
  \hline
  1 & No & 154 &  35 & 189 \\ 
  2 & Sí &  78 &  18 &  96 \\ 
  3 & Total & 232 &  53 & 285 \\ 
   \hline
\end{tabular}
\end{table}

Vemos como son muy pocos, exactamente 18 persona de las 290 que trabajan
actualmente y este trabajo tiene que ver con la carrera seleccionada el
día del examen.

\begin{Shaded}
\begin{Highlighting}[]
\NormalTok{result }\OtherTok{\textless{}{-}} \FunctionTok{pivotr}\NormalTok{(}
\NormalTok{  data, }
  \AttributeTok{cvars =} \FunctionTok{c}\NormalTok{(}
    \StringTok{"tipo\_de\_colegio\_que\_estudia\_o\_estudio"}\NormalTok{, }
    \StringTok{"escoja\_una\_de\_las\_siguientes\_razones\_por\_la\_que\_va\_a\_estudiar"}
\NormalTok{  ), }
  \AttributeTok{tabsort =} \StringTok{"desc(Publico)"}\NormalTok{, }
  \AttributeTok{nr =} \ConstantTok{Inf}
\NormalTok{)}
\CommentTok{\# summary()}
\CommentTok{\# result$tab\_freq \%\textgreater{}\% xtable()}
\end{Highlighting}
\end{Shaded}

\begin{table}[ht]
\centering
\begin{tabular}{rlrrr}
  \hline
 & escoja\_una\_de\_las\_siguientes\_razones\_por\_la\_que\_va\_a\_estudiar & Privado & Publico & Total \\ 
  \hline
1 & Así lo hacen mis parientes, mis amigos, mi novio (a) &   0 &   1 &   1 \\ 
  2 & Así lo quieren mis papás &   1 &   3 &   4 \\ 
  3 & Con los estudios que tengo no podré conseguir un buen trabajo &   7 &   9 &  16 \\ 
  4 & Es normal &   1 &   6 &   7 \\ 
  5 & Me interesa estudiar &  37 &  71 & 108 \\ 
  6 & Me va a dar prestigio. &   3 &   7 &  10 \\ 
  7 & No tengo otra cosa que hacer &   3 &   9 &  12 \\ 
  8 & Quiero ganar bien cuando sea profesionista &   2 &  12 &  14 \\ 
  9 & Quiero superarme &  27 &  65 &  92 \\ 
  10 & Quiero vivir más independiente de mi familia &   5 &  15 &  20 \\ 
  11 & Un (a) maestro (a) me dijo que siguiera estudiando &   1 &   5 &   6 \\ 
  12 & Total &  87 & 203 & 290 \\ 
   \hline
\end{tabular}
\end{table}

Otra comparación que resulta llamativa es comparar el tipo de colegio en
el que estudio o estudia y la razón por la cual quiere entrar a una
universidad.

Son muy parecidos en el conteo de los estudiantes que quieren superarse
y les interesa estudiar que salieron de colegios públicos.

\end{document}
